
\documentclass[11pt]{article}


\usepackage{extramarks} % Required for headers and footers
\usepackage[usenames,dvipsnames]{color} % Required for custom colors
\usepackage{graphicx} % Required to insert images
\usepackage{listings} % Required for insertion of code
\usepackage{courier} % Required for the courier font

%----------------------------------------------------------------------------------------
%	DOCUMENT STRUCTURE COMMANDS
%	Skip this unless you know what you're doing
%----------------------------------------------------------------------------------------

% Header and footer for when a page split occurs within a problem environment
\newcommand{\enterProblemHeader}[1]{
\nobreak\extramarks{#1}{#1 continued on next page\ldots}\nobreak
\nobreak\extramarks{#1 (continued)}{#1 continued on next page\ldots}\nobreak
}

% Header and footer for when a page split occurs between problem environments
\newcommand{\exitProblemHeader}[1]{
\nobreak\extramarks{#1 (continued)}{#1 continued on next page\ldots}\nobreak
\nobreak\extramarks{#1}{}\nobreak
}

\setcounter{secnumdepth}{0} % Removes default section numbers
\newcounter{homeworkProblemCounter} % Creates a counter to keep track of the number of problems

\newcommand{\homeworkProblemName}{}
\newenvironment{homeworkProblem}[1][Problem \arabic{homeworkProblemCounter}]{ % Makes a new environment called homeworkProblem which takes 1 argument (custom name) but the default is "Problem #"
\stepcounter{homeworkProblemCounter} % Increase counter for number of problems
\renewcommand{\homeworkProblemName}{#1} % Assign \homeworkProblemName the name of the problem
\section{\homeworkProblemName} % Make a section in the document with the custom problem count
\enterProblemHeader{\homeworkProblemName} % Header and footer within the environment
}{
\exitProblemHeader{\homeworkProblemName} % Header and footer after the environment
}

\newcommand{\problemAnswer}[1]{ % Defines the problem answer command with the content as the only argument
\noindent\framebox[\columnwidth][c]{\begin{minipage}{0.98\columnwidth}#1\end{minipage}} % Makes the box around the problem answer and puts the content inside
}

\newcommand{\homeworkSectionName}{}
\newenvironment{homeworkSection}[1]{ % New environment for sections within homework problems, takes 1 argument - the name of the section
\renewcommand{\homeworkSectionName}{#1} % Assign \homeworkSectionName to the name of the section from the environment argument
\subsection{\homeworkSectionName} % Make a subsection with the custom name of the subsection
\enterProblemHeader{\homeworkProblemName\ [\homeworkSectionName]} % Header and footer within the environment
}{
\enterProblemHeader{\homeworkProblemName} % Header and footer after the environment
}

%----------------------------------------------------------------------------------------
%	NAME AND CLASS SECTION
%----------------------------------------------------------------------------------------

\newcommand{\hmwkTitle}{LMOCH: a model checker for Lustre} % Assignment title
\newcommand{\hmwkDueDate}{Sunday, 14 December, 2014} % Due date
\newcommand{\hmwkClass}{Synchronous Systems} % Course/class
\newcommand{\hmwkAuthorName}{Mario ALVAREZ-PICALLO} % Your name

%----------------------------------------------------------------------------------------
%	TITLE PAGE
%----------------------------------------------------------------------------------------

\title{
\textmd{\textbf{\hmwkTitle}}\\
\normalsize\vspace{0.1in}\small{Due\ on\ \hmwkDueDate}\\
}

\author{\textbf{\hmwkAuthorName}}

\setlength{\parskip}{5pt}

\begin{document}

\maketitle

\section{1. Scope of the project}
LMOCH is a model checker for simple Lustre programs by $k$-induction. Its purpose is to automatically
check that a given Lustre node satisfies a certain predicate, given by the output of that node, which
must be a boolean.

This model checker works by translating the AST of a Lustre program into a set of
logical formulae which represent the semantics of said node as a function of time. Said formulae are
then fed into the SMT solver Alt-Ergo-Zero in order to determine if the desired predicate holds.

\section{2. Structure of the model checker}
The model checker LMOCH consists of two components: a translator and a verifier.

The translator, implemented in the file 'src/translator.ml', is rather simple. At its core are two mutually
recursive functions, 'expression\_to\_formula' and 'expression\_to\_term'. These functions convert a Lustre
expression into an AEZ formula or term respectively, and a set of additional logical formulae that specify
the behavior of generated auxiliary variables.

The translation process, although relatively straightforward, is rather complete, and allows for the handling
of calls to different nodes. This is done through embedding the equations for the called node into the 
equationsof the caller node. For this process to be sound, it is necessary to ensure that if a node calls
another
node more than once, the variables introduced by the equations of the callee don't overlap with each other,
while at the same time ensuring that calls are preserved through time, i.e. that the same call in different
instants of time doesn't introduce new fresh variables for each instant. This has been the source of subtle bugs
through development.

After the Lustre program has been converted to a set of logical formulae parameterized by time, the verifier,
defined in the file 'src/verifier.ml', runs a simple $k$-induction algorithm on the resulting formulae. This
algorithm is rather simple: first, the verifier checks that the base case holds, i.e. that the node has the
desired property in the first $k$ instants of time. If at this point AEZ finds a contradiction in the model,
this means that the checked property does not hold.

On the other hand, if the base case succeeds, the verifier proceeds to check the inductive step, by asserting
the node equations and the desired property for all instants $n - 1 \dots n - k$ and from this attempts to infer
the property for instant $k$. If the property holds, then the process concludes and the verifier reports success
. However, if AEZ fails to prove the desired property, the verifier increments the depth $k$ of the induction
process and starts from the base case again, until the depth $k$ excedes the bound specified by the user.

As an optimization, the verifier implements the path compression strategy described in Hagen and Tinelli. 
This optimization adds an extra assumption to the inductive step of the verification process, that asserts that
the last $k$ steps taken by the node don't repeat the same state. The rationale behind this is that in order
to prove the property it suffices to check all loop-free paths. Mechanically, this is implemented by asserting
that for all $j < i \leq k$ there exists a certain variable $x$ belonging to the inputs or the internal
variables of the node such that $x_i \neq x_j$. 

\section{3. Limitations}
One of the main limitations of LMOCH is inherited from the SMT solver: since AEZ is unable to handle 
floating-point numbers, LMOCH is unable to perform verification on Lustre code that uses them. 

A more fundamental limitation, however, is that LMOCH can only prove properties that are expressible in Lustre,
which limits its usefulness to arithmetic or logical properties that depend only on the last $k$ steps of a 
trace. It is therefore impossible to prove liveness properties using LMOCH. This, however, should not be a
crippling limitation, since Lustre is commonly used in embedded, real-time control software, which usually
requires safety properties, but not liveness ones.

\section{4. Compilation details}
The provided software has been compiled and tested on a \nobreak{Ubuntu 14.04 64 bits} laptop, using the standard
OCaml compiler \nobreak{version 4.02.1, 64 bits.}

\end{document}
